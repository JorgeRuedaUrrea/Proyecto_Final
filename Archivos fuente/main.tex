\documentclass{article}
\usepackage[utf8]{inputenc}
\usepackage[spanish]{babel}
\usepackage{listings}
\usepackage{graphicx}
\graphicspath{ {images/} }
\usepackage{cite}

\begin{document}

\begin{titlepage}
    \begin{center}
        \vspace*{1cm}
            
        \Huge
        \textbf{Ideas proyecto final}
            
        \vspace{0.5cm}
        \LARGE
        INFORMATICA 2
            
        \vspace{1.5cm}
            
        \textbf{Jorge Enrique Rueda Urrea}
            
        \vfill
            
        \vspace{0.8cm}
            
        \Large
        Despartamento de Ingeniería Electrónica y Telecomunicaciones\\
        Universidad de Antioquia\\
        Medellín\\
        Marzo de 2021
            
    \end{center}
\end{titlepage}

\tableofcontents
\newpage
\section{Ideas generales}\label{intro}

Sera un juego de plataformas en el cual el objetivo principal sera sobrepasar los distintos obstaculos y enemigos que iran apareciendo a medida que transcurra el tiempo, cada nivel incluira un "jefe final" el cual sera una prueba mas dura que todo lo transcurrido hasta ese momento.\newline
El personaje principal sera un oso grizzly, este sera el que el usuario va a controlar.\newline\newline
Los enemigos seran humanos los cuales disparan balas en distintas direcciones pero siendo siempre enfocados hacia nuestro personaje principal.\newline\newline
Las plataformas a usar constaran de superficies planas,superficies con pinchos y superficies irregulares todas estas distribuidas al rededor del mapa de juego, en estas plataformas podran posarse tanto nuestro personaje principal como los enemigos quienes tendran un movimiento automatico.\newline\newline



\section{Como hacer realidad las ideas} \label{contenido}
Para poder llevar a cabo este proyecto sera necesario tener claridad en todos los temas que se vayan viendo a medida que transcurre el semestre , pero inicialmente se tiene claridad de varios aspectos.\newline\newline
La fisica newtoniana jugara un papel fundamental ya que el movimiento de nuestros personajes y proyectiles estaran regidos por las leyes de movimiento.\newline
Para hacer el codigo mas amable para mi como programador se implementara la gran mayoria del codigo en funciones las cuales se iran invocando segun sea el caso.\newline\newline



\section{modo de juego} \label{contenido}
El modo se juego sera individual o en equipo (max 2 jugadores), para esto se hara unicamente uso de el teclado.\newline
Las teclas de movimiento seran asignadas segun el jugador.\newline\newline
Para el jugador #1 las teclas de movimiento seran las flechas del teclado (arriba, abajo, izquierda, derecha) y para lanzar los artefactos para destruir los enemigos se hara uso de la tecla p.\newline\newline
Para el jugador #2 las teclas de movimiento seran w, s, a, d las cuales corresponderan respectivamente a los movimioentos arriba, abajo, izquierda y derecha, para lanzar artefactos hara uso de la tecla Ctrl.\newline\newline
Cuando el modo seleccionado es el de dos jugadores, el jugador #1 sera el encargado de atacar a los enemigos, mientras que el jugador #2 sera solo un apoyo, es decir, con su tecla de lanzamiento curara a su compañero pero a los enemigos no les hara daño,por lo que la sinergia entre jugadores sera dde vital importancia



\end{document}
